\documentclass[12pt, a4paper, oneside]{article}
\usepackage{arial}
\renewcommand{\familydefault}{\sfdefault}
\usepackage[T1]{fontenc}
\usepackage[polish]{babel}
\usepackage[utf8]{inputenc}
\usepackage{lmodern}
\usepackage[left=2cm,right=2cm,top=2cm,bottom=2cm]{geometry}
\selectlanguage{polish}

\begin{document}
\section{Portal do zamawiania jedzenia}
\indent\indent Portal zawiera ofertę gastronomiczną lokali zlokalizowanych na terenie Wrocławia. Każdy z~nich oferuje sprzedaż produktów wraz z dowozem w~określonych godzinach, dla określonych, za pomocą kodu pocztowego, lokalizacji. Pełna oferta portalu zawiera punkty czynne w ciągu dnia, nocą, a~także całodobowe. Zamówienia internetowe można składać maksymalnie do trzydziestu minut przed zamknięciem lokalu.
\section{System}
\indent\indent Główną funkcjonalnością systemu jest składanie zamówień przez użytkowników. Po złożeniu zamówienia otrzymuje ono status 'Przyjęte', następnie restaurator może zmienić status na 'W realizacji', a~potem 'Zrealizowane'. Przy każdym lokalu pokazywana jest cena dostawy wyliczona jako iloczyn stawki kilometrowej lokalu i odległości do klienta.
\subsection{Administrator systemu}
\indent\indent W systemie istnieje konto administratora, który ma uprawnienia do dodawania, usuwania, a~także pełnej edycji oferty lokali, tj. dodawanie produktów, usuwanie produktów, zmiana cen, ustawianie potraw jako nieaktywne do końca dnia, ustawienie wcześniejszych godzin zamknięcia wybranego dnia, natychmiastowe zamknięcie lokalu. Ma on też możliwość edycji oraz usuwania użytkowników.
\subsection{Klient}
\indent\indent Zarejestrowany użytkownik posiada w bazie danych określone przy rejestracji dane dostawy oraz kontaktowe, które może edytować samodzielnie w~dowolnym czasie z poziomu swojego konta. Klient po zalogowaniu posiada dostęp do historii zamówień z ostatniego roku.
\subsection{Restaurator}
\indent\indent Restaurator, ze specjalnego konta ma możliwość zmiany statusu zamówienia.
\subsection{Lokale}
\indent\indent Każdy lokal przedstawiany jest klientowi przy pomocy danych, takich jak:
\begin{itemize}
\item nazwa,
\item godziny pracy,
\item ocena klientów,
\item lokalizacja,
\item kontakt,
\item oferowane produkty, które są specyfikowane przez: cenę, nazwę, krótki opis.
\end{itemize}
\subsection{Składanie zamówienia}
\begin{enumerate}
\item Logowanie do systemu.
\item Dodanie zamówienia do koszyka (możliwość sortowania restauracji).
\item Potwierdzenie danych, wybranie preferowanego czasu dostawy (jak najszybciej, za 2 godziny, etc.), wybranie formy płatności.
\item Po upłynięciu 15 minut od ustawienia statusu 'Zrealizowane', użytkownik może wystawić ocenę restauracji.
\end{enumerate}
\end{document}